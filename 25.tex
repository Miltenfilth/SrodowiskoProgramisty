\documentclass[12pt,a4paper]{article}

% ustawienia marginesu
\usepackage[left=1.6in,right=.8in,top=1.5in,bottom=1.5in]{geometry}

% polskie reguły dzielenia wyrazów itd
\usepackage{polski}

% polskie znaki zakodowane w UTF8
\usepackage[utf8]{inputenc}

% automatyczne generowanie odnośników w plikach PDF
\usepackage[pdftex,linkbordercolor={0 0.9 1}]{hyperref}

% pakiety matematyczne
\usepackage{amsthm,amsmath,amsfonts,amssymb,mathrsfs}

% ładne składanie odnośników do stron www
\usepackage{url}

% rozbudowane możliwości wypunktowań
\usepackage{enumerate}

% możliwość dodawania plików graficznych
\usepackage{graphicx} 

%%% definicje twierdzeń itd :)
\newtheorem{tw}{Twierdzenie}[section]
\newtheorem{stw}[tw]{Stwierdzenie}
\newtheorem{fakt}[tw]{Fakt}
\newtheorem{lemat}[tw]{Lemat}

\theoremstyle{definition}
\newtheorem{df}[tw]{Definicja}
\newtheorem{ex}[tw]{Przykład}
\newtheorem{uw}[tw]{Uwaga}
\newtheorem{wn}[tw]{Wniosek}
\newtheorem{zad}{Zadanie}

% oznaczenia zbiorów liczbowych
\DeclareMathOperator{\R}{\mathbb{R}}
\DeclareMathOperator{\Z}{\mathbb{Z}}
\DeclareMathOperator{\N}{\mathbb{N}}
\DeclareMathOperator{\Q}{\mathbb{Q}}


% wartość bezwzględna, norma, iloczyn skalarny, nośnik, rozpięcie przestrzeni...
\providecommand{\abs}[1]{\left\lvert#1\right\rvert}
\providecommand{\var}[1]{\operatorname{var}(#1)}

% fajne nagłówki i stopki na stronie
\usepackage{fancyhdr}
\pagestyle{fancy}
\fancyhf{}
\fancyfoot[R]{\textbf{\thepage}}
\fancyhead[L]{\small\sffamily \nouppercase{\leftmark}}
\renewcommand{\headrulewidth}{0.4pt} 
\renewcommand{\footrulewidth}{0.4pt}




\title{Funkcje ciągłe i różniczkowalne}
\author{Maciej Stankiewicz}

\date{30 listopada 2010}

\begin{document}
\maketitle
\tableofcontents


\section{Funkcje ciągłe}
	\begin{df}
		(funkcja ciągła). Niech $f : (a,b) \rightarrow \R$, oraz niech $x_0 \in (a,b)$. %
		Mówimy, że funkcja $f$ jest ciągła w punkcie $x_0$ wtedy i tylko wtedy, gdy:\\

	$\forall_{\epsilon > 0} \exists_{ \delta > 0} \forall x \in (a,b) |x - x_0| < \delta \Rightarrow |f(x) - f(x_0)| < \epsilon$.
	\end{df}
	\begin{ex}
		Wielomiany, funkcje trygonometrycznem wykładnicze, logarytmiczne są ciągłe w każdym punkcje swojej dziedziny.
	\end{ex}
	\begin{ex}
		Funkcja $f$ dana wzorem:\\
	\begin{displaymath} 
		f(x)= \left\{\begin{array}{lll}
		x+1 & \textrm{dla} & x \neq 0 \\
		0 & \textrm{dla} & x=0
		\end{array} \right.
 	\end{displaymath}
		Jest ciągła w każdym punkcie poza $x_0 = 0$.\\
		Niech $\Q$ oznacza zbiór wszystkich liczb wymiernych.\\

	\end{ex}
	\begin{ex}
		Funkcja $f$ dana wzorem:\\
		\begin{displaymath} 
			f(x)= \left\{\begin{array}{lll}
			0 & \textrm{dla} & x \in \Q \\
			1 & \textrm{dla} & x \notin \Q
			\end{array} \right.
 		\end{displaymath}
		nie jest ciągła w żadnym punkcie.\\
	\end{ex}
	\begin{ex}
		Funkcja $f$ dana wzorem:\\
		\begin{displaymath} 
			f(x)= \left\{\begin{array}{lll}
			0 & \textrm{dla} & x \in \Q \\
			x & \textrm{dla} & x \notin \Q
			\end{array} \right.
 		\end{displaymath}
		jest ciagla w punkcie $x_0 = 0$, ale nie jest ciągła w pozostałych punktach dziedziny.\\
	\end{ex}
	\begin{zad}
		Udowodnij prawdziwość podanych przykładów.\\
	\end{zad}
	\begin{df}
		Jeśli funkcja $f:  A \rightarrow \R$ jest ciągła w każdym punkcie swojej dziedziny $A$ to mówimy krótko, że jest ciągła.


		Poniższe twierdzenie zbiera podstawowe włsności zbioru funkcji ciągłych.\\
	\end{df}
	\begin{tw}
		Niech funkcje $f,g: R \rightarrow \R$ będą ciągłe, oraz niech $\alpha, \beta \in \R$.\\
		Wtedy funkcje:
		\begin{enumerate}
			\item[a)] $h_1(x)=\alpha*f(x)+\beta*g(x)$,
			\item[b)] $h_2(x)=f(x)*g(x)$,
			\item[c)] $h_3(x)=\frac{f(x)}{g(x)}$ (o ile $g(x)\neq0$ dla dowolnego $x\in \R$),
			\item[d)] $h_4(x)=f(g(x))$,
		\end{enumerate}
		są ciągłe.


		Następne twierdzenie zwane powszechnie "własnością Darboux" lub twierdzeniem o wartości pośredniej ma liczne praktyczne zastosowania. Mówi ono o tym, że jeśli funkcja ciągła przyjmuje jakieś dwie wartości, to przy odpowiednich założeniach co do dziedziny, przyjmuje  też wszystkie wartości pośrednie. Możemy sobie to łatwo wyobrazić na przykładzie funkcji, która opisuje zmianę temperatury w czasie. Jeśli o 7:00 było $-1\,^{\circ}\mathrm{C}$ a o 9:00 było $2\,^{\circ}\mathrm{C}$ to zapewne gdzieś między 7:00 a 9:00 był taki moment, że temperatura wynosiła dokładnie $0\,^{\circ}\mathrm{C}$.

	\end{tw}
	\begin{tw}
		Niech $f:[a,b] \rightarrow \R$ ciągła, oraz niech $f(a) \neq f(b)$. Wtedy dla dowolnego $y_0 \in conv\{f(a),f(b)\}$ istnieje $x_0 \in [a,b]$ takie, że $f(x_0)=y_0$.
	\end{tw}
\section{Różniczkowalność}

	\begin{df}
		Niech $f:(a,b) \rightarrow \R, x_0 \in (a,b)$ oraz $f$ ciągła w otoczeniu punktu $x_0$. Jeśli istnieje granica:\\
		\begin{displaymath}
			\lim_{x \to x_0} \frac{f(x)-f(x_0)}{x-x_0}
		\end{displaymath}

i jest skończonam to oznaczamy ją przez $f'(x_0)$ i nazywamy pochodną funkcji $f$ w punkcie $x_0$.
	\end{df}
	\begin{df}
		Jeśli funkcja $f$ posiada pochodną w każdym punkcie swojej dziedziny, to mówimy, że $f$ jest różniczkowalna. Istnieje wtedy funkcja $f'$, która każdemu punktowi z dziedziny funkcji $f$ przyporządkowuje wartość pochodnej funkcji $f$ w tym punkcie.
	\end{df}
	\begin{ex}
		Wielomiany, funkcje trygonometryczne, wykładnicze, kogarytmiczne są różniczkowalne w każdym punkcie dziedziny.
	\end{ex}
	\begin{ex}
		Funkcja $f(x)=|x|$ jest ciągła, ale nie posiada pochodnej w punkcie $x_0=0$.	
	\end{ex}
	\begin{tw}
		Niech $f: [a,b] \rightarrow \R$ ciągła i różniczkowalna na $(a,b)$. Dodatkowo niech $f'(x) \neq 0 dla x\in (a,b)$, oraz niech $m=min_{x\in [a,b]} f(x), M=max_{x \in [a,b]}f(x).$\\
Wtedy na pewno $f(a)=m, f(b)=M$ lub $f(a)=M$ i $f(b)=m$.
	\end{tw}
		

\end{document}
